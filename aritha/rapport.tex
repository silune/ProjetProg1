\documentclass[11pt,a4paper]{article}
\usepackage[french]{babel}
\usepackage{listings}
\usepackage[T1]{fontenc}
\usepackage{verbatim}
\usepackage[margin=1.2in]{geometry}
\usepackage{amsmath}
\usepackage{amsfonts}
\usepackage{amssymb}

\newcommand{\dd}{\text{d}•}

\pagenumbering{arabic}

\begin{document}


\begin{titlepage}

\vspace{1cm}

	\begin{center}

		\huge{\textbf{Compilateur d'expressions arithmétiques "\textit{aritha}"}}

		\vspace{5mm} %espace vertical de 5mm

		\begin{large}

		Projet Prog 1

		\vspace{5mm}

		Simon Corbard

		\end{large}

	\end{center}

\vfill

\tableofcontents

\end{titlepage}

\section{Définitions / Choix}

\subsection{Lexemes}

Les lexemes sont définis par :


\begin{verbatim}
lexeme := | L_bra | R_bra | Int_fun | Float_fun
          | Add_int | Sub_int | Mul_int | Div | Mod
          | Add_float | Sub_float | Mul_float
          | Int of string | Float of string
\end{verbatim}

J'ai fait le choix de différencier les valeurs numériques flottantes et entières au lieu d'utiliser un lexeme "\verb|Var of string|". Je n'ai pas differencié, dans le lexer, les opérateurs unaires "\verb|+|" et "\verb|-|" et les opérateurs binaires représentés par les même caractères. Je n'ai pas implémenter de lexeme "\verb|Function of string|" pour implémenter les fonctions \verb|int| et \verb|float| (ou possiblement d'autres) car cela rendais mon lexer plus complexe, ce qui n'est pas nécessaire pour la gestion des expressions regardées.\\
Rq : Les fonctions \verb|int| et \verb|float| sont typées, en particulier \verb|int(0)| n'est pas valide.\\
Rq : J'accèpte dans mon lexer les notations pour les flottants : \verb|4.| et \verb|.04|.

\subsection{AST / TAST}

Les arbres syntaxiques abstraits sont typés et définis par :

\begin{verbatim}
tast := | INTFUN of tast | FLOATFUN of tast
        | ADDI of tast * tast | SUBI of tast * tast | MULI of tast * tast
        | NEGI of tast | NEGF of tast
        | DIVI of tast * tast | MODI of tast * tast
        | ADDF of tast * tast | SUBF of tast * tast | MULF of tast * tast
        | INT of string | FLOAT of string
\end{verbatim}

Le parseur transforme la liste de lexemes en arbre syntaxique, en respectant l'ordre de priorité : \verb|(float, int) -> (*, *., /, %) -> (+, +., -, -.)|. Le type est vérifié après génération de l'arbre par un deuxième parcours, et lors de la création des noeuds \verb|NEGI| et \verb|NEGF| dans lesquels le type deriere l'opération unaire est inféré. Sont aussi vérifiés à cette étape : le bon parenthèsage de l'expression et le bon appel des fonctions.

\section{Problèmes rencontrés}

\subsection{Gestion des flottants avec x86-64}

Il n'y a pas, en x86-64, de fonction pour push le contenu d'un registre sur la pile, il a donc fallu le faire à la main : le registre \verb|%rsp| indique la position de la tête de la pile, on retire donc 8 octets au pointeur \verb|0(%rsp)| et on \verb|movq| le registre souhaité à cette adresse (et symétriquement pour pop un élément de la pile).

Pour gérer la pile plus proprement, je déplace manuellement la base de la pile (dont l'adresse est stockée dans \verb|%rbp|) à la tête de la pile, de sorte à avoir une pile vide lors de l'execution de \verb|main|. A la fin de l'execution je repositionne la base et la tête sur leurs positions initiales afin que l'execution du programme n'est aucune influence sur la pile.

\subsection{Division Euclidienne}

Lors des test sur la division Euclidienne et le modulo sur des entiers négatifs, les résultats était incohérent. En réalité, \verb|idivq reg| réalise la division Euclidienne du registre \verb|reg| par \verb|%rdx::%rax|. Cependant il ne suffit pas de mettre \verb|$0| dans \verb|%rdx| et le dénominateur dans \verb|%rax| car le signe du denominateur (situé dans le bit le plus à gauche du registre) est perdu. La commande \verb|cqto| permet de faire la conversion, et réalise donc les divisions et modulo attendus en C.

\subsection{Parser}

Comme j'ai réalisé mon Parseur à la main, il était assez difficile de faire un parseur le plus général possible, notamment pour la gestion d'opérations unaires (\verb|+(exp)| et \verb|-(exp)|) et leurs distinctions avec les opérateurs binaires ou les constructions de constantes. Notamment mon premier parseur réalisait l'assosiativité à droite et j'ai du le repenser complètement pour la faire à gauche.

\section{Bonus}

\subsection{Autres opérateurs}

J'ai ajouté les opérateurs : factorielle, puissance entière d'entiers, division de flottants. Représentés par les expressions respectivent : \verb|exp!|, \verb|exp^exp|, \verb|exp/.exp|. La nouvelle règle de priorité est donc :\\
\verb|(float, int) -> (!, ^) -> (*, /, *., /., %) -> (+, +., -, -.)|

En assembleur, la gestion des opérateurs factorielle et puissances nécessitent des appels de fonctions, ces fonctions sont donc inscrites en bas de chaques codes assembleurs renvoyés par \verb|aritha| mais il serrait plus satisfaisant qu'elle apparaissent seuelement si nécéssaire.

\subsection{Générateur d'expressions arithmétiques}

Dans le but de tester \verb|aritha| sur un plus grand nombre d'essais j'ai implémenter un générateur d'expressions arithmétiques valides qui produit un code Ocaml évaluant l'expression, et le fichier \verb|.exp| associé à l'expression.\\
Rq: ces tests ne permettent pas de s'assurer que les expressions invalides sont bien rejettées par \verb|aritha|\\
\noindent Utilisation:\\

\noindent \verb|./randomExpGenerator fileName operationNumber typeReturned maximumConstant|\\

\noindent fileName : nom des fichiers rendus en sortie\\
operationNumber : nombre d'opérations utilisées dans l'expression\\
typeReturned : int ou float\\
maximumConstant : valeure maximum des constantes générées aléatoirement\\
Rq : les deux derniers arguments sont optionels avec par défault : int 100\\
Example d'utilisation réalisant les tests successifs de \verb|aritha| et le code Ocaml :
\begin{verbatim}
./randomExpGenerator testExp 10 int 10 &&
ocamlc testExp.answer.ml -o testExp.answer && ./testExp.answer &&
../aritha testExp.test.exp &&
gcc -no-pie testExp.test.s -o testExp.test && ./testExp.test
\end{verbatim}

\end{document}
